\section{单项选择题}
\begin{enumerate}
	\item 不能使$\frac{\partial^2 u }{\partial x \partial y } = 2x - y $的解为\emptychoice .
		\begin{choice}(1)
			\choice $ u = x^2 y - \frac{1}{2} x y^2 $
			\choice $ u = x^2 y - \frac{1}{2} x y^2 -5 $
			\choice $ u = x^2 y - \frac{1}{2} x y^2 + e^x + e^y -5 $
			\choice $ u = x^2 y - \frac{1}{2} x y^2 + e^{x+y} -5 $
		\end{choice}
	\item 二元函数$z=f(x,y)$的两个偏导数存在且$\frac{\partial z}{\partial x}>0$,$\frac{\partial z}{\partial y}>0$,则\emptychoice .
		\begin{choice}(1)
			\choice 当$\Delta x > 0$且$\Delta y > 0$时,$\Delta z<0$
			\choice 当$\Delta x > 0$且$\Delta y > 0$时,$\Delta z>0$
			\choice 当$\Delta x > 0$且$\Delta y > 0$时,$\dd z > 0$且$\Delta z>0$
			\choice 当$\Delta x > 0$且$\Delta y > 0$时,$\dd z > 0$但$\Delta z$不一定大于零
		\end{choice}
	\item 若函数$z=f(x,y)$在点$P_0(x_0,y_0)$处的两个偏导数存在,则\emptychoice .
		\begin{choice}(1)
			\choice $z=f(x,y)$在点$P_0$处连续
			\choice $z=f(x,y)$在点$P_0$处存在全微分
			\choice $\begin{cases}
z=f(x,y)\\
y=y_0
\end{cases}$在点$P_0$处连续
			\choice 以上都不对
		\end{choice}
	\item 已知函数$f(x+y,x-y)=x^2-y^2$,则$\frac{\partial f(x,y)}{\partial x}+\frac{\partial f(x,y)}{\partial y} = $\emptychoice .
		\begin{choice}(4)
			\choice $2x-2y$
			\choice $2x+2y$
			\choice $ x-y $
			\choice $ x+y $
		\end{choice}
	\item 二元函数在$ (x_0,y_0) $的极限存在是函数在该点连续的\emptychoice .
		\begin{choice}(4)
			\choice 充分条件
			\choice 必要条件
			\choice 充要条件
			\choice 都不是
		\end{choice}
	\item 设$ u=f(x+y,xz) $有二阶偏导数,则$ \frac{\partial^2 u}{\partial x \partial z}= $\emptychoice .
		\begin{choice}(1)
			\choice $ f'_2 + xf''_{11} +zf''_{12} + xf''_{12} $
			\choice $ f'_2 + xf''_{21} + xzf''_{22} $
			\choice $ xf''_{21}+ xzf''_{22} $
			\choice $ xf''_{12} + f'_2 + xzf''_{22} $
		\end{choice}
	\item 函数$ y=\ln(-x-y) $的定义域是\emptychoice .
		\begin{choice}(1)
			\choice $ \{ (x,y)|x<0,y<0 \} $
			\choice $ \{ (x,y)|x+y\leqslant 0 \} $
			\choice $ \{ (x,y)|x+y<0 \} $
			\choice $ \{ (x,y)|x,y\in \mathbb{R} \} $
		\end{choice}
	\item 偏导数存在是全微分存在的\emptychoice 条件.
		\begin{choice}(4)
			\choice 充分
			\choice 必要
			\choice 充分必要
			\choice 以上皆不对
		\end{choice}
	\item 二元函数的两个偏导数存在是该函数连续的\emptychoice 条件.
		\begin{choice}(4)
			\choice 充分
			\choice 必要
			\choice 充分必要
			\choice 以上皆不是
		\end{choice}
\end{enumerate}
\section{填空题}
\begin{enumerate}
	\item 函数$z=\arcsin \frac{x}{2} + \arcsin \frac{y}{3} $的定义域为\blank .
	\item $\lim\limits_{x\rightarrow 0,y\rightarrow 2} \frac{\sin xy}{x} = $\blank .
	\item $ \lim\limits_{x\to 0,y\to 1} \frac{1- xy}{x^2+y^2} =$\blank .
	\item $ \lim\limits_{x\to 0,y\to 0} \frac{2-\sqrt{xy+4}}{xy} = $\blank .
	\item 函数$ f(x,y)=\frac{\sqrt{4x-y^2}}{\ln(1-x^2-y^2)} $的定义域为\blank[9em].
	\item 已知$ z=x^{y^2} $,则$ \dd z = $\blank[9em].
\end{enumerate}
\section{计算题}
\begin{enumerate}
	\item 求$ z=x^3y-xy^3 $的一阶偏导数.\vspace{4.7cm}
	\item 求$ z=\sqrt{\ln(xy)} $的一阶偏导数.\vspace{4.7cm}
	\item 求$ u=x^{\frac{y}{z}} $的一阶偏导数.\vspace{4.7cm}
	\item 求$ u=\arctan(x-y)^z $的一阶偏导数.\vspace{4.7cm}
	\item 求$ z=xy+\frac{x}{y} $的全微分.\vspace{4.7cm}
	\item 求$ w=x^{yz} $的全微分.\vspace{4.7cm}
	\item 求$ \lim\limits_{x\to\infty,y\to\infty} \left( \frac{xy}{x^2+y^2} \right)^{x^2} $.\vspace{4.7cm}
	\item 设$ f(x,y)=e^{-y}\cdot\sin(2x+y) $,求$ f_x(\frac{\pi}{4},0) $和$ f_y(\frac{\pi}{4},0) $.\vspace{4.7cm}
	\item 求$ z=x^4+y^4-4x^2y^2 $的二阶偏导数.\vspace{4.7cm}
	\item 求$ z=y^x $的二阶偏导数.\vspace{4.7cm}
	\item 求$ z=f(x,\frac{x}{y}) $的二阶偏导数.\vspace{4.7cm}
	\item 设$ z=\sin\frac{x}{y}\cdot\cos\frac{y}{x} $,求$ \dd z\big|_{(1,1)} $.\vspace{4.7cm}
\end{enumerate}
\section{单项选择题解答}
\begin{enumerate}
	\item D\\
		解析:D选项,$\frac{\partial u }{\partial x } = 2xy - \frac{1}{2}y^2 + e^{x+y} $,$\frac{\partial^2 u }{\partial x \partial y } = 2x - y + e^{x+y} $.
	\item D\\
		解析:$\Delta z = \frac{\partial z}{\partial x} \Delta x + \frac{\partial z}{\partial y} \Delta y + o(\rho)$,$ \dd z = \frac{\partial z}{\partial x} \dd x + \frac{\partial z}{\partial y} \dd y $. $o(\rho)$的符号不能确定,所以选D.
	\item C\\
		解析:A选项,二元函数连续与偏导数存在没有任何关系.\\
B选项:二元函数偏导数连续时一定有全微分.\\
C选项:偏导数的思想即为固定一个变量时,函数因变量对另外一个自变量的导数,此时函数是一元函数,对一元函数来说,可导一定连续.
	\item D\\
		解析:首先计算出$f(x,y)$的表达式,$f(x+y,x-y)=x^2-y^2 = (x+y)(x-y)\Rightarrow f(x,y)=xy$,所以$\frac{\partial f(x,y)}{\partial x}+\frac{\partial f(x,y)}{\partial y} = x+y$.
	\item B\\
		解析:根据连续的定义可知,当某点处极限与函数值相同时函数在该点连续。所以连续一定有极限,但有极限不一定连续.
	\item D\\
		解析:$ \frac{\partial u}{\partial x} = f'_1 + zf'_2 $,$ \frac{\partial^2 u}{\partial x \partial z}= \frac{\partial}{\partial z}\left(f'_1 + zf'_2\right) = xf''_{12} + f'_2 + xzf''_{22}$.
	\item C\\
		解析:$ \ln $要求其参数大于零.
	\item B\\
		解析:见课本73页定理1.
	\item D\\
		解析:二元函数的偏导数存在与连续没有关系,见课本68-69页.
\end{enumerate}
\section{填空题解答}
\begin{enumerate}
	\item \underline{$ \{ (x,y) | -2\leqslant x \leqslant 2, -3\leqslant y \leqslant 3 \} $}\\
		解析:$ \arcsin x $的定义是域是$ [-1,1] $.
	\item \underline{2}\\
		解析:$\lim\limits_{x\to 0,y\to 2} \frac{\sin xy}{x} = \lim\limits_{x\to 0,y\to 2} \frac{\sin xy}{xy} \cdot \lim\limits_{y\to 2}y = 2 $.
	\item \underline{1}\\
		解析:直接代入.
	\item \underline{$-\frac{1}{4}$}\\
		解析:$ \lim\limits_{x\to 0,y\to 0} \frac{2-\sqrt{xy+4}}{xy} = \lim\limits_{u\to 0} \frac{2-\sqrt{u+4}}{u} = \lim\limits_{u\to 0} \frac{(2-\sqrt{u+4})(2+\sqrt{u+4})}{u(2+\sqrt{u+4})} = \lim\limits_{u\to 0} \frac{-u}{u(2+\sqrt{u+4})} = -\frac{1}{4} $.
	\item \underline{$ \{ (x,y)|0 < x^2+y^2 < 1 \text{且} y^2\leqslant 4x \} $}\\
		解析:有三个条件需要考虑,根号下表达式应大于等于零,自然对数函数的参数应大于零,分母不能为零。
	\item \underline{$ x^{y^2-1}y^2\dd x + 2x^{y^2}y\ln x \dd y $}\\
		解析:求出$ z $对$ x,y$的偏导数后按全微分公式写出答案.
\end{enumerate}
\section{计算题解答}
\begin{enumerate}
	\item 解:$ \frac{\partial z}{\partial x}=3x^2y-y^3 $,$ \frac{\partial z}{\partial y}=x^3-3xy^2 $.
	\item 解:$ \frac{\partial z}{\partial x}=\frac{1}{2x\sqrt{\ln(xy)}} $,$ \frac{\partial z}{\partial y}=\frac{1}{2y\sqrt{\ln(xy)}} $.
	\item 解:$ \frac{\partial u}{\partial x}=\frac{y}{z} x^{\frac{y}{z}-1} $,$ \frac{\partial u}{\partial y}=\frac{1}{z} x^{\frac{y}{z}}\ln x $,$ \frac{\partial u}{\partial z}=-\frac{y}{z^2} x^{\frac{y}{z}}\ln x $.
	\item 解:$ \frac{\partial u}{\partial x}=\frac{z(x-y)^{z-1}}{1+(x-y)^{2z}} $,$ \frac{\partial u}{\partial y}=-\frac{z(x-y)^{z-1}}{1+(x-y)^{2z}} $,$ \frac{\partial u}{\partial z}=\frac{(x-y)^{z}\ln(x-y)}{1+(x-y)^{2z}} $.
	\item 解:$ \dd z = (y+\frac{1}{y})\dd x + (x-\frac{x}{y^2})\dd y $.
	\item 解:$ \dd w = yzx^{yz-1}\dd x + x^{yz}z\ln x \dd y + x^{yz}y\ln x \dd z $.
	\item 解:因为$ 0 \leqslant \left| \left( \frac{xy}{x^2+y^2} \right)^{x^2} \right| \leqslant \left| \left( \frac{xy}{2xy} \right)^{x^2} \right| \leqslant \left( \frac{1}{2} \right)^{x^2}$,且$ \lim\limits_{x\to\infty,y\to\infty} \left( \frac{1}{2} \right)^{x^2} =0 $,所以根据夹逼准则,有$ \lim\limits_{x\to\infty,y\to\infty} \left( \frac{xy}{x^2+y^2} \right)^{x^2} =0 $.
	\item 解:$ f_x(x,y)=2e^{-y}\cdot \cos (2x+y) $,$ f_y(x,y)=-e^{-y}\cdot \sin (2x+y) + e^{-y}\cdot \cos (2x+y) $,代入得$ f_x(\frac{\pi}{4},0)=0 $,$ f_y(\frac{\pi}{4},0)=-1 $.
	\item 解:$ \frac{\partial z}{\partial x}=4 x^{3}-8 x y^{2} $,$\frac{\partial z}{\partial y}=4 y^{3}-8 x^{2} y$,$\frac{\partial^{2} z}{\partial x^{2}}=12 x^{2}-8 y^{2}$,$\frac{2 x}{2 y^{2}}=12 y^{2}-8 x^{2}$,$\frac{\partial^{2} z}{\partial x \partial y}=-16 x y $.
	\item 解:$\frac{\partial z}{\partial x}=y^{x} \cdot \ln y $,$ \frac{\partial z}{\partial y}=x \cdot y^{x-1}$,$\frac{\partial^{2} z}{\partial x^{2}}=y^{x}(\ln y)^{2}$,$\frac{\partial^{2} z}{\partial y^{2}}=x(x-1) y^{x-2}$,$\frac{\partial^{2} z}{\partial x \partial y}=(x \ln y+1) y^{x-1}$.
	\item 解:$$\small
\begin{aligned}
	\frac{\partial z}{\partial x}&=\frac{\partial f}{\partial x}+\frac{\partial f}{\partial u}\cdot \frac{\partial y}{\partial x}=f_{1}^{\prime}+\frac{1}{y}f_{2}^{\prime}\\
	\frac{\partial z}{\partial y}&=\frac{\partial f}{\partial u}\cdot \frac{\partial y}{\partial y}=-\frac{x}{y^2}f_{2}^{\prime}\\
	\frac{\partial ^2z}{\partial x^2}&=\frac{\partial}{\partial x}\left( f_{1}^{\prime}+\frac{1}{y}f_{2}^{\prime} \right) =\frac{\partial f_{1}^{\prime}}{\partial x}+\frac{1}{y}\frac{\partial f_{2}^{\prime}}{\partial x}=f_{11}^{\prime\prime}+\frac{1}{y}f_{12}^{\prime\prime}+\frac{1}{y}\left( f_{21}^{\prime\prime}+\frac{1}{y}f_{22}^{\prime\prime} \right) =f_{11}^{\prime\prime}+\frac{2}{y}f_{12}^{\prime\prime}+\frac{1}{y^2}f_{22}^{\prime\prime}\\
	\frac{\partial ^2z}{\partial x\partial y}&=\frac{\partial}{\partial y}\left( f_{1}^{\prime}+\frac{1}{y}f_{2}^{\prime} \right) =\frac{\partial f_{1}^{\prime}}{\partial y}+\left( -\frac{1}{y^2} \right) \cdot f_{2}^{\prime}+\frac{1}{y}\frac{\partial f_{2}^{\prime}}{\partial y}=-\frac{x}{y^2}f_{12}^{\prime\prime}-\frac{1}{y^2}f_{2}^{\prime}+\frac{1}{y}\cdot \left( -\frac{x}{y^\prime}f_{22}^{\prime\prime} \right) =-\frac{x}{y^2}f_{12}^{\prime\prime}-\frac{x}{y^3}f_{22}^{\prime\prime}-\frac{1}{y^2}f_{2}^{\prime}\\
	\frac{\partial ^2z}{\partial y^2}&=\frac{\partial}{\partial y}\left( -\frac{x}{y^2}f_{2}^{\prime} \right) =-x\left( -\frac{2}{y^3}f_{2}^{\prime}+\frac{1}{y^2}\frac{\partial f_{2}^{\prime}}{\partial y} \right) =\frac{2x}{y^3}f_{2}^{\prime}-\frac{x}{y^2}\left( -\frac{x}{y^2}f_{22}^{\prime\prime} \right) =\frac{x^2}{y^4}f_{22}^{\prime\prime}+\frac{2x}{y^3}f_{2}^{\prime}\\
\end{aligned}
$$
	\item 解:\[\dd z=\left(\frac{1}{y} \cos \frac{x}{y} \cos \frac{y}{x}+\frac{y}{x^{2}} \sin \frac{x}{y} \sin \frac{y}{x}\right) \dd x -\left(\frac{x}{y^{2}} \cos \frac{x}{y} \cos \frac{y}{x}+\frac{1}{x} \sin \frac{x}{y} \sin \frac{y}{x}\right) \dd y\]
\[ \dd z \big|_{(1, 1)}=\dd x -\dd y\]
\end{enumerate}
